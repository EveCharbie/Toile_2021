
%%%%%%%%%%%%%%%%%%%%%%%%%%%%%%%%%%%%%%%%%%%%%%%%%%%%%%%%%%%%%%%%%%%%%%%%%%%%%%%%
%2345678901234567890123456789012345678901234567890123456789012345678901234567890
%        1         2         3         4         5         6         7         8

\documentclass[preprint,12pt]{elsarticle}  % Comment this line out if you need a4paper

\usepackage[utf8]{inputenc}
\usepackage{amsmath}
\usepackage{bm}
\usepackage{amssymb}
\usepackage{graphicx}
\usepackage{booktabs}
\usepackage{mathrsfs}
\usepackage{amsbsy}
% \usepackage{caption}
\usepackage[center]{caption}
\usepackage{float}
\usepackage{subcaption}
\usepackage{varwidth}
\usepackage{algorithm}
\usepackage{yhmath}
\usepackage{tabularx}
\usepackage{rotating}
\usepackage[table]{xcolor}
\definecolor{grey}{rgb}{0.9,0.9,0.9}
\usepackage[noend]{algpseudocode}
\usepackage{dsfont}
\usepackage{environ}
\usepackage[framemethod=TikZ]{mdframed}
\usepackage{multirow}
\usepackage{algorithm}
\usepackage{accents}\newcommand\addtag{\refstepcounter{equation}\tag{\theequation}}
\usepackage{etoolbox}
\usepackage{graphicx}
\usepackage{siunitx}
\usepackage{rotating}
\usepackage{enumerate}
\usepackage{hyperref}
\usepackage{relsize}
\usepackage{gensymb}
\usepackage{color}
\usepackage{comment}
\newcommand{\comFB}[1]{~\linebreak\noindent\colorbox{yellow}{\parbox{\dimexpr\columnwidth-1\fboxsep}{[FB: #1]}}}
\newcommand{\comEC}[1]{~\linebreak\noindent\colorbox{green}{\parbox{\dimexpr\columnwidth-1\fboxsep}{[EC: #1]}}}
\newcommand{\comTEMPO}[1]{~\linebreak\noindent\colorbox{gray}{\parbox{\dimexpr\columnwidth-1\fboxsep}{[#1]}}}
%%% Customized commands
%%% example
\newcommand\myNewCommand{\mathbf{\texttt{my new command}}}

\newcommand{\mvec}[1]{\bm{#1}}
\newcommand{\dmvec}[1]{\dot{\mvec{{#1}}}}
\newcommand{\ddmvec}[1]{\ddot{\mvec{{#1}}}}

\newcommand{\st}{\text{s.t.}}
\newcommand{\g}{\mvec{g}}
\newcommand{\q}{\mvec{q}}
\newcommand{\dq}{\dmvec{q}}
\newcommand{\ddq}{\ddmvec{q}} 
\title{\LARGE \bf
Optimal Control as a Tool for Innovation in Aerial Twisting on Trampoline
}
\author{Eve Charbonneau\textsuperscript{a,*}, François Bailly\textsuperscript{a}, Loane Danès\textsuperscript{b} and Mickaël Begon\textsuperscript{a}% <-this % stops a space
\thanks{\textsuperscript{a}\,Laboratoire de Simulation et Modélisation du Mouvement, Faculté de Médecine, Université de Montréal, Laval, QC, Canada}%
\thanks{\textsuperscript{b}\,AgroParisTech, Paris, France}%
\thanks{\textsuperscript{*}\,eve.charbonneau.1@umontreal.ca}
}

%%% User commands

\usepackage{pdfrender}
\DeclareRobustCommand*{\pmbb}[1]{%
  \textpdfrender{
    TextRenderingMode=Stroke,
    LineWidth=.1pt,
  }{#1}%
}
\NewEnviron{comeq}{%
\par\vspace{0ex}
\begin{mdframed}[outerlinewidth=0.5,leftmargin=10,rightmargin=-10pt,backgroundcolor=white,hidealllines=true,leftline=true,
innertopmargin=0pt,splittopskip=0, skipbelow=\baselineskip, innerbottommargin=0pt%
skipabove=0ex]%
\vspace{-0ex}\hspace{0pt}\textit{Proof:}%
\itshape
\begin{equation*} 
\begin{split}
\BODY
\end{split}
\end{equation*}
\end{mdframed}
}
\newcommand{\pd}[2]{\frac{\partial #1}{\partial #2}}
\def\abs{\operatorname{abs}}
\def\argmax{\operatornamewithlimits{arg\,max}}
\def\argmin{\operatornamewithlimits{arg\,min}}
\def\diag{\operatorname{Diag}}
\newcommand{\eqRef}[1]{(\ref{#1})}
\newcommand{\dbtilde}[1]{\accentset{\approx}{#1}}
\hypersetup{
    colorlinks=true,
    linkcolor=blue,
    filecolor=magenta,      
    urlcolor=blue,
}
\hyphenpenalty=100000

\begin{document}


\thispagestyle{empty}
\pagestyle{empty}

\begin{abstract}
% Aerial twisting techniques have gained coaches' interest since the flight time is part of the score in trampoline. 
% As these techniques are not intuitive, computer simulation has been a relevant tool to explore a variety of techniques. 
% Unfortunately, twisting somersaults were mainly simulated using arm abd/adduction only. 
% Our objective was to explore more complex but still anatomically feasible arm techniques to find innovative and robust twisting techniques.
% The twist rotation was maximized in a straight backward somersault performed by a model including abd/adduction with and without % change in plane of elevation. 
% This optimal control problem was solved by direct multiple-shooting. 
% A multi-start approach (n=440) was used to find a series of locally optimal performances. 
% The robustness of the selected techniques was assessed by adding noise on the arm kinematics and then reoptimizing to mimic the athlete's adaptation to kinematic changes. 
% Three innovative techniques which generate approximately three pure aerial twists were selected. 
% Techniques found by optimization share a highly twisting strategy consisting in moving the arm in a plane formed by the twisting and angular momentum axes.
% A mechanical analysis is presented to help coaches include this strategy in their professional practice. 
To Do
\end{abstract}


\maketitle

\textbf{Keywords -- Trampoline bed, optimal control,  somersaults, model simulation, coaching}

\section{Introduction}\label{sec:introduction}
\comEC{Ajuster les temps au passé toujours}

% C'est important d'étudier les take-off au trampoline
The addition of flight time to trampoline score in 2010 \cite{Committee2010} has changed coaches' perception of ideal propulsion technique.
Achieving high somersault velocity while maintaining important jumping height has become more important than ever.
The complexity of this task lies in generating linear and angular momenta at the same time.
The first one is maximal when the force generated by the trampoline force is aligned with the center of mass (CoM) of the athlete, while the second is maximal perpendicularly, i.e. when the force is orthogonal to the vector feet-CoM, maximizing the moment arm. 
Therefore a compromise at each instant during the propulsion phase.
\comEC{schéma de ca vue que c'est plutôt important comme concept ou emphase dans la discussion ?}
% Even though coaches are able to recognize visually the quality of a take-off, it stays challenging to formulate feedback for athletes because kinematic errors might originate from different sources.
% The athlete might not perform the desired motor patern due to strength limitations, active flexibility limitations, lack of coordination  or bad timing.
Without a deeper insight into the biomechanics of the take-off, it is challenging to determine the precise timeline of actions that should be executed by the athlete during the contact phase.
\comFB{notion d'optimalite ?}


% Le meilleur model de toile c'est celui de Jacques2008
The first step to analyze athlete-trampoline interaction is to investigate the force applied by the trampoline on the athlete.
However, measuring this force is challenging due to the movement of the contact point.
In order to quantify it, some studies used force transducers under the trampoline supports \cite{jacques2008determining, ando1987biomechanical, hennig1988loads}, insoles \cite{glitsch1992pressure} or tri-axial accelerometers placed on the athlete \cite{eager2012characterisation}.
Other studies used a combination of modeling and video recording to estimate the force generated by the trampoline \cite{vaughan1980kinetic, blajer2001modeling, zuo2016finite, burke2015mechanics}.
These indirect techniques leverage the relationship between the trampoline deformation and the force generation.
These methods have the advantage to simply model the trampoline's force-displacement relationship with polynomial expressions.
However, in order to calibrate the parameters of these models, drop and load tests are usually needed.
These tests get more dangerous as the trampoline deformation increases, especially horizontally, therefore it becomes risky to match the level of bed depression achieved by athletes (up to 1 meter).
\comEC{ref je crois}
\comFB{ Bizarre de parler d'une depression horizontale. -> pour ca que je dis deformation a la place}
In these circumstances, force-displacement relationships could not be fully confirmed with experimental data. 
It could only be hypothesized that the tendency would be preserved throughout larger bed deformations.
Finally, one study used more complex modeling enabling to estimate the force from static measurements on springs and bed separately \cite{jacques2008determining}.
This technique has the advantage to reliably estimate vertical and horizontal forces at the same time while providing a safer data collection.


% C'est important de tenir en compte la force vertical et horizontale dans analyses et modelisation
Until now, trampoline forces were studied with an emphasis on the vertical component.
However, the trampoline bed and spring system has the ability to deform both vertically and horizontally giving rise to a three dimensional force.
For non-twisting somersault, the component of the force in the frontal plane must be null to avoid lateral displacements during the flight phase.\comFB{J'ai du mal a visualiser sans forcer. Un schema ?}
The same logic does not apply for the sagittal plane because the athlete's motion \comFB{which part of the motion ?}
generating somersault happen in this plane.
Indeed, sagittal plane motions enlarging the contribution to angular momentum include three strategies: \textit{i)} shift the CoM, \textit{ii)} change the orientation of the force at the application point or \textit{iii)} move the application point.
Forward somersaulting take-off strategies were investigated to analyze CoM and feet trajectories to assess the biomechanical strategies leading to generation of angular and linear momentum \cite{lephartatiner}.
\comFB{Pourquoi est-ce qu'on passe aux forward somersault ? -> pcq il a juste fait ca et je trouvais important de le mentionner ?}
Strategies involving solely vertical force on the trampoline were qualified as not viable due to the loss of height and horizontal displacement they imply \cite{lephartatiner}.
\comFB{Cela est sense justifier le modele 3d du trampo ? -> oui /  Tu veux dire que la seule strategie viable est ii) ? -> Non}
Therefore, it seems necessary to analyze both horizontal and vertical bed work.


% Le controle optimal est un bon outil pour étudier les take-offs
Optimal control is a relevant tool to study sport techniques because it helps (in)validate the techniques currently used \cite{charbonneau2020optimal}.
Indeed, it allows to identify biomechanical strategies and provide advises to the sport community accordingly.
Jumping motion on compliant surfaces have rarely been studied by means of numerical optimization \cite{cheng2008role, burke2015mechanics}.
When they were, only one type of acrobatic skill was considered, i.e. backward or forward somersaults.
However, trampolining is composed of both types of skills, typically combined in alternation to compose the required 10-skills routine. 
Therefore, studying the transition between skills is more relevant than single skills to the sport community.
Moreover, the optimization must include the contact phase and the aerial phase of the studied skills, as both are interdependent, e.g., a bad posture during take off might impact negatively the beginning of the flight phase. 
\comEC{Mal dit je crois: la seule étude qui a fait optim trampo a juste optimisé toile -> max momentum at take-off}


% A la fois les modeles de toile et d'athlete doivent être assez complexes pour représenter la réalité 
\comEC{Revoir au complet !!!}
To draw biomechanical conclusions from strategies found by optimization, the model must be validated with experimental data, like it was the case for the trampoline model used in \cite{jacques2008determining}.
\comEC{Ca dit que c'est le meilleur qui existe mais qu'il reste du travail a faire :(}
\comFB{je ne vois pas apparaître la notion de travail à faire !}
Similarly, the athlete's model used in \cite{burke2015mechanics} showed a great level of accordance with experimental data.
% It was torque actuated, however torque values were non linearly bounded to physiological values.
The biggest challenge associated with this model was it's lack of degrees of freedom (DoF).
Indeed, it was mentionned \comFB{where ?} that the addition of one shoulder and one thoracic DoF would be beneficial.


The first aim of this study was to find bed work techniques \comFB{bed work techniques ? -> pas une bonne expression ?} generating somersaults at maximal height in trampoline. 
A secondary objective was to identify the limiting factors by providing a biomechanical analysis of the underlying strategies.
\comFB{pas très clair}
% We hypothesized that for the same number of somersault completed, the height of the avatar taking advantage of 2D force would be grater than for the avatar propelled by 1D force only. 


\comEC{Étude de sensibilité littérature? (force, flex épaules, morphologie)}
% The studies addressing trampoline jumping optimization offered a vertical representation of the contact force (referred to as \textit{1D force}), however the force applied on the athlete's feet has three components. 
% 2D modeling of the trampoline contact force, referred to as \textit{2D force}, would allow to fully use the trampoline deformation to generate somersault of maximal height.
\comFB{À la fin de ton intro, on ne sait pas vraiment ce que tu vas faire. On ne sait pas si tu vas retravailler un modèle de trampo, etc.}


\section{Methods}\label{sec:methods}
\subsection{Skeletal Model}\label{subsec:2a}
% To assess the improvements brought by 3D motions of the arms, two models were created with Spatial Vector and Rigid-Body Dynamics Software~\cite{featherstone2014rigid} to represent the body of the trampolinist.
% In accordance with the literature, the first one included a free floating base and one DoF at each arm for abd/adduction movements, whereas the second one included two DoFs at each arm to add the possibility to change the plane of elevation (Fig.~\ref{fig:Model1}).
% Both models were composed of three rigid bodies connected with inelastic joints: the root segment including the head, the trunk and lower limbs, the right and left upper limbs with extended elbows.

% \begin{figure}[h!]
% \centering
% \includegraphics[width=0.5\linewidth]{figures/Model_Final_2.png}
% \caption{Model definition with 10 degrees-of-freedom: translations of the root ($q_{1-3}$), rotations of the root ($q_{4-6}$), right arm  and left arm abd/adduction ($q_{7-8}$), right arm and left arm change in plane of elevation ($q_{9-10}$).}
% \label{fig:Model1}
% \end{figure}

% Both models were torque driven at the shoulders, i.e., two controls for the 2D model and four controls for the 3D one.
% Inertial parameters of the bodies were estimated from 95 anthropometric measurements of one subject (female, 21~years old, 65~kg, 158~cm) in line with Yeadon's anthropometric model~\cite{yeadon1990simulation}.
To Do


\subsection{Trampoline Model}\label{subsec:2b}
To Do

\subsection{Formulation of the optimization problem}\label{subsec:2c}
% The main objective of the optimization problem was to maximize the number of twists generated by the arm technique (first term in Eq.~\eqref{eq:ocp}). 
% Some penalties were added to generate realistic techniques, namely minimizing the efforts required to execute the movement (second term in Eq.~\eqref{eq:ocp}), minimizing the somersault rotation velocity at take-off (third term in Eq.~\eqref{eq:ocp}) and minimizing the length of the hands trajectories (fourth term in Eq.~\eqref{eq:ocp}). 
% The cost function $\mathcal{J}$ to be minimized was:

% \begin{eqnarray}\label{eq:ocp}
% \mathcal{J} = -\alpha q_6(T)  + \sum_{i=1}^{nb_{\tau}}  \beta_i \int_0^T \tau_{i}^2 dt  + \gamma \dot{q_4}(0) + \delta \sum_{k=1}^{2} \int_{\wideparen{P_k}} ds,
% \end{eqnarray}
% \comEC{Mickael: t not in J -> non, not in J est-ce que tu veux que je le précise ?}
% where $t$ is the flight time ranging from $0$ to $T$ (take-off to landing), $\alpha,\beta_i,\gamma$ and $\delta$ are weighting coefficients, $\tau_{i}$ is the control of the $i^{th}$ DoF and $P_k$ is the $k^{th}$ hand path in space.
% For the 2D model (resp. the 3D one) $nb_{\tau} = 2$ (resp. $4$).
% The $\int_{~\wideparen{.}}$ symbol in the last term of Eq.~\eqref{eq:ocp} denotes a line integral along a path, which, in our case, represents the distance traveled by each hand.
% We looked for suitable compromises between the four objective terms by manually tuning $\left\lbrace\alpha,\beta_i,\gamma,\delta\right\rbrace$ to obtain highly twisting yet realistic techniques.
% The expertise of EC (first author and national trampoline coach) guided us through this process.
% Four types of problems (termed as 2D-PHP, 2D-UHP, 3D-PHP and 3D-UHP) were solved using the 2D or 3D model and by penalizing (PHP) or not (UHP) the hand path.
% The following coefficients were used depending on the type of problem :  $\alpha=100000$, $\gamma=100$, $\beta_{1,2}=0.01$, $\beta_{3,4}=1$ for 3D model and $\delta$ is receptively set to $1$, $0$, $100000$ and $0$ for 2D-PHP, 2D-UHP, 3D-PHP and 3D-UHP.
To Do


\subsection{Multi-start approach}\label{subsec:2d}
% Since the OCP was solved using a gradient-based algorithm, the optimal solution is a local minimum which satisfies the exit criterion.
% Therefore optimal techniques are likely to differ depending on the provided initial solution~\cite{huchez2015local}.
% In the scope of the present study, local minima were an opportunity to explore efficient techniques which were not globally optimal, but still interesting compromises from the sport's perspective. 
% A multi-start strategy was used to generate several solutions, as in~\cite{huchez2016differences}.
% Each OCP (2D-PHP, 2D-UHP, 3D-PHP and 3D-UHP) was given 440 initial solutions using combinations of the following parameters: 
% \begin{itemize}
% \item Twist time history ($q_6(t)$) linearly increasing from $0$ to $[2,3,4,5]$ rotations
% \item Number of shooting nodes $N \in \mathcal{N} = \left\lbrace 295...305\right\rbrace \subset \mathbb{N}$
% \item Random arm elevation ($q_{7,8}(t)$)
% \item Random arm torques ($\tau_{1-4}(t)$)
% \end{itemize}


% The best solutions of 2D-UHP and 3D-UHP (hand path unpenalized, thus exhibiting the highest complexity) were selected for further analysis.
% The locally optimal solutions of 2D-PHP ($360 \pm 10\degree$) and locally optimal solutions of 3D-PHP (twist $> 1080\degree$) were reviewed separately by EC (first author and national trampoline coach).
% Among them, we selected one 2D-PHP solution for its similarity with currently used strategies in trampoline and three 3D-PHP solutions for their innovation potential.
% The latest were chosen because of their minimal complexity and efforts, which is relevant for athletes and coaches.
% In total, six techniques were further analyzed in Sec.~\ref{subsec:3a}.


\subsection{Biomechanical analysis of the solutions}\label{subsec:2e}
% \comEC{US, je vais le passer à antidote il ne drvait plus rester de fautes de s/z... Il est écrit qu'ils acceptent les deux donc je propose US}
% To evaluate the effect of techniques complexity on performance, least mean square error was evaluated for a linear correlation between the number of twists generated and the length of the hand path.
% To analyze arm movements generating aerial twists, we introduce the ``\textit{best tilting plane}'' (BTP) which is the plane formed by the twisting axis and by the angular momentum vector (constant in aerial phase).
% In a frame fixed to the body, this plane is rotating (consequently to the twist rotation).
% The angle ($\phi$) between the BTP and the plane of arm motions was calculated at each node for all optimal techniques.
% The orientation of the BTP was obtained from the cross product between the twisting and the angular momentum axes and arm movement direction were evaluated with finite differences from their kinematics expressed in the frame of the body.
% \comEC{Non pas regardé qdot pcq pour connaitre la direction il faudrait que je multiple ap run petit dt et ca revient au meme selon moi}

% To discriminate parts of the skill including aerial twist strategies, the following criteria were used:
% \begin{itemize}
% \item the arm is moving ($\dot{q}_{7,8}(t) > 90\degree/s$)
% \item the twist acceleration is positive ($\ddot{q}_{6}(t) > 0\degree/s$)
% \item the arm is not aligned with the body (the angle between the arm and the body is in the range $[10,170]\degree$)
% \end{itemize}








\section{Results}\label{sec:results}
\subsection{Analysis optimal solutions}\label{subsec:3a}
To Do

\subsection{Impact of objectives weighting}\label{subsec:3b}
To Do


\section{Discussion}\label{sec:discussion}
% The main objective of this study was to generate and analyze innovative and robust twisting techniques to highlight arm strategies that trampolinists could integrate into their practice.
% The main findings are that \textit{(i)} aerial twist performance is linearly correlated to the complexity of arm trajectories, \textit{(ii)} moving the arms in the so-called ``\textit{best tilting plane}'' is an efficient strategy to generate twist rotation and \textit{(iii)} for similar twist performance, 3D techniques are simpler and require less efforts than 2D techniques. 

1) position du CoM au cours du contact \cite{lephartatiner}
2) translation horizontale du a la force verticale (!?!?)
3) positionnement en quittant la toile


\subsection{To be determined}\label{subsec:4a}
To Do


\subsection{Limitations}\label{subsec:4e}
The trampoline used in \cite{} to determine the constants of the model is a 6mm x 4mm is a model of trampoline that is not used any more in FIG competitions.
No damping coefficients (due to air friction through the trampoline web).
Point contact between the feet and the trampoline. (No ankle, no region of depression of the bed)
During the contact phases, the feet are considered attached to the trampoline (No friction coeff has been used to model contact).

\section{Conclusion}\label{sec:conclusion}
Main findings
% This research shows that athletes could gain in performance by incorporating 3D motions to their techniques, especially by moving their segments in the best tilting plane (i.e., plane formed by the twist and angular momentum axes).
% We provided three innovative and simple 3D arm techniques for a triple twist backward somersault robust to kinematic perturbations. 
% The selected 3D arm techniques were less complex, required less efforts and had better potential for landing than the equivalent 2D technique, establishing the superiority of 3D arm techniques.
% With the comprehension brought by this work, coaches could better guide their future athletes throughout the learning of 3D arm techniques.


\setcitestyle{number}
\bibliographystyle{elsarticle-num} 
\bibliography{biblio}

\clearpage

\begingroup
\let\clearpage\relax 
\onecolumn 
\section{Appendix}\label{sec:appendix}
\subsection{The best tilting plane.}\label{subsec:5a}
To Do

\endgroup




\end{document}
