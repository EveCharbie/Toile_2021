\subsection{Skeletal Model}\label{subsec:2a}
% To assess the improvements brought by 3D motions of the arms, two models were created with Spatial Vector and Rigid-Body Dynamics Software~\cite{featherstone2014rigid} to represent the body of the trampolinist.
% In accordance with the literature, the first one included a free floating base and one DoF at each arm for abd/adduction movements, whereas the second one included two DoFs at each arm to add the possibility to change the plane of elevation (Fig.~\ref{fig:Model1}).
% Both models were composed of three rigid bodies connected with inelastic joints: the root segment including the head, the trunk and lower limbs, the right and left upper limbs with extended elbows.

% \begin{figure}[h!]
% \centering
% \includegraphics[width=0.5\linewidth]{figures/Model_Final_2.png}
% \caption{Model definition with 10 degrees-of-freedom: translations of the root ($q_{1-3}$), rotations of the root ($q_{4-6}$), right arm  and left arm abd/adduction ($q_{7-8}$), right arm and left arm change in plane of elevation ($q_{9-10}$).}
% \label{fig:Model1}
% \end{figure}

% Both models were torque driven at the shoulders, i.e., two controls for the 2D model and four controls for the 3D one.
% Inertial parameters of the bodies were estimated from 95 anthropometric measurements of one subject (female, 21~years old, 65~kg, 158~cm) in line with Yeadon's anthropometric model~\cite{yeadon1990simulation}.
To Do


\subsection{Trampoline Model}\label{subsec:2b}
To Do

\subsection{Formulation of the optimization problem}\label{subsec:2c}
% The main objective of the optimization problem was to maximize the number of twists generated by the arm technique (first term in Eq.~\eqref{eq:ocp}). 
% Some penalties were added to generate realistic techniques, namely minimizing the efforts required to execute the movement (second term in Eq.~\eqref{eq:ocp}), minimizing the somersault rotation velocity at take-off (third term in Eq.~\eqref{eq:ocp}) and minimizing the length of the hands trajectories (fourth term in Eq.~\eqref{eq:ocp}). 
% The cost function $\mathcal{J}$ to be minimized was:

% \begin{eqnarray}\label{eq:ocp}
% \mathcal{J} = -\alpha q_6(T)  + \sum_{i=1}^{nb_{\tau}}  \beta_i \int_0^T \tau_{i}^2 dt  + \gamma \dot{q_4}(0) + \delta \sum_{k=1}^{2} \int_{\wideparen{P_k}} ds,
% \end{eqnarray}
% \comEC{Mickael: t not in J -> non, not in J est-ce que tu veux que je le précise ?}
% where $t$ is the flight time ranging from $0$ to $T$ (take-off to landing), $\alpha,\beta_i,\gamma$ and $\delta$ are weighting coefficients, $\tau_{i}$ is the control of the $i^{th}$ DoF and $P_k$ is the $k^{th}$ hand path in space.
% For the 2D model (resp. the 3D one) $nb_{\tau} = 2$ (resp. $4$).
% The $\int_{~\wideparen{.}}$ symbol in the last term of Eq.~\eqref{eq:ocp} denotes a line integral along a path, which, in our case, represents the distance traveled by each hand.
% We looked for suitable compromises between the four objective terms by manually tuning $\left\lbrace\alpha,\beta_i,\gamma,\delta\right\rbrace$ to obtain highly twisting yet realistic techniques.
% The expertise of EC (first author and national trampoline coach) guided us through this process.
% Four types of problems (termed as 2D-PHP, 2D-UHP, 3D-PHP and 3D-UHP) were solved using the 2D or 3D model and by penalizing (PHP) or not (UHP) the hand path.
% The following coefficients were used depending on the type of problem :  $\alpha=100000$, $\gamma=100$, $\beta_{1,2}=0.01$, $\beta_{3,4}=1$ for 3D model and $\delta$ is receptively set to $1$, $0$, $100000$ and $0$ for 2D-PHP, 2D-UHP, 3D-PHP and 3D-UHP.
To Do


\subsection{Multi-start approach}\label{subsec:2d}
% Since the OCP was solved using a gradient-based algorithm, the optimal solution is a local minimum which satisfies the exit criterion.
% Therefore optimal techniques are likely to differ depending on the provided initial solution~\cite{huchez2015local}.
% In the scope of the present study, local minima were an opportunity to explore efficient techniques which were not globally optimal, but still interesting compromises from the sport's perspective. 
% A multi-start strategy was used to generate several solutions, as in~\cite{huchez2016differences}.
% Each OCP (2D-PHP, 2D-UHP, 3D-PHP and 3D-UHP) was given 440 initial solutions using combinations of the following parameters: 
% \begin{itemize}
% \item Twist time history ($q_6(t)$) linearly increasing from $0$ to $[2,3,4,5]$ rotations
% \item Number of shooting nodes $N \in \mathcal{N} = \left\lbrace 295...305\right\rbrace \subset \mathbb{N}$
% \item Random arm elevation ($q_{7,8}(t)$)
% \item Random arm torques ($\tau_{1-4}(t)$)
% \end{itemize}


% The best solutions of 2D-UHP and 3D-UHP (hand path unpenalized, thus exhibiting the highest complexity) were selected for further analysis.
% The locally optimal solutions of 2D-PHP ($360 \pm 10\degree$) and locally optimal solutions of 3D-PHP (twist $> 1080\degree$) were reviewed separately by EC (first author and national trampoline coach).
% Among them, we selected one 2D-PHP solution for its similarity with currently used strategies in trampoline and three 3D-PHP solutions for their innovation potential.
% The latest were chosen because of their minimal complexity and efforts, which is relevant for athletes and coaches.
% In total, six techniques were further analyzed in Sec.~\ref{subsec:3a}.


\subsection{Biomechanical analysis of the solutions}\label{subsec:2e}
% \comEC{US, je vais le passer à antidote il ne drvait plus rester de fautes de s/z... Il est écrit qu'ils acceptent les deux donc je propose US}
% To evaluate the effect of techniques complexity on performance, least mean square error was evaluated for a linear correlation between the number of twists generated and the length of the hand path.
% To analyze arm movements generating aerial twists, we introduce the ``\textit{best tilting plane}'' (BTP) which is the plane formed by the twisting axis and by the angular momentum vector (constant in aerial phase).
% In a frame fixed to the body, this plane is rotating (consequently to the twist rotation).
% The angle ($\phi$) between the BTP and the plane of arm motions was calculated at each node for all optimal techniques.
% The orientation of the BTP was obtained from the cross product between the twisting and the angular momentum axes and arm movement direction were evaluated with finite differences from their kinematics expressed in the frame of the body.
% \comEC{Non pas regardé qdot pcq pour connaitre la direction il faudrait que je multiple ap run petit dt et ca revient au meme selon moi}

% To discriminate parts of the skill including aerial twist strategies, the following criteria were used:
% \begin{itemize}
% \item the arm is moving ($\dot{q}_{7,8}(t) > 90\degree/s$)
% \item the twist acceleration is positive ($\ddot{q}_{6}(t) > 0\degree/s$)
% \item the arm is not aligned with the body (the angle between the arm and the body is in the range $[10,170]\degree$)
% \end{itemize}






