% The main objective of this study was to generate and analyze innovative and robust twisting techniques to highlight arm strategies that trampolinists could integrate into their practice.
% The main findings are that \textit{(i)} aerial twist performance is linearly correlated to the complexity of arm trajectories, \textit{(ii)} moving the arms in the so-called ``\textit{best tilting plane}'' is an efficient strategy to generate twist rotation and \textit{(iii)} for similar twist performance, 3D techniques are simpler and require less efforts than 2D techniques. 

1) position du CoM au cours du contact \cite{lephartatiner}
2) translation horizontale du a la force verticale (!?!?)
3) positionnement en quittant la toile


\subsection{To be determined}\label{subsec:4a}
To Do


\subsection{Limitations}\label{subsec:4e}
The trampoline used in \cite{} to determine the constants of the model is a 6mm x 4mm is a model of trampoline that is not used any more in FIG competitions.
No damping coefficients (due to air friction through the trampoline web).
Point contact between the feet and the trampoline. (No ankle, no region of depression of the bed)
During the contact phases, the feet are considered attached to the trampoline (No friction coeff has been used to model contact).