% To get higher scores in competitions, trampolinists enhance their 10-skill routines by increasing the amount of somersault and/or twist rotations.
% There exist three strategies to create twist rotation during a somersault, namely contact, cat and aerial strategies~\cite{frohlich1980physics}. 
% Contact twists are generated when the athlete is in contact with the trampoline bed, by inducing an initial twisting angular momentum along the longitudinal axis (termed as the twisting axis). It will be conserved throughout the aerial phase of the skill, until the landing. 
% While twist velocity is desired during the skill, it is not recommended when landing onto the trampoline bed, as it leads to less balanced landings~\cite{sanders1995effect}.
% Cat twists are generated by rotating segments opposite to the desired twist rotation~\cite{yeadon1999learning}, such as hula-hoop motions.
% This strategy gets more efficient as the radii of rotation of the segments increase. 
% However, in straight somersaults, such hips and trunk postures are penalized by judges through a stylistic score, discarding their extensive use among athletes.
% Aerial twists are created after take-off by segments asymmetries aiming to bring the twisting and the angular momentum axes closer together by tilting the body~\cite{dullin2016twisting}.
% This strategy often involves asymmetrical arm motions~\cite{yeadon1999learning}. 
% While trampolinists can use a combination of these three strategies, the greatest contribution to twist usually comes from aerial strategies~\cite{yeadon1993biomechanics4}.


% In pure aerial strategies, there is no twisting rotation velocity at take-off, before landing, it is possible to stop twisting by un-tilting the body, i.e., bringing the twist and angular momentum axes perpendicular to each other~\cite{yeadon2013limits}. 
% Consequently, and in contrast to contact twists, this strategy has the advantage to allow athletes to land with more balance~\cite{sanders1995effect} --- an advantage which has led trampoline coaches to promote aerial twisting.
% However, the mechanics of aerial twisting is complex because it is subject to non-linear effects (i.e., Coriolis and centrifugal forces) leading to non-trivial control strategies, which require a precise timing of the sequences of actions.
% Coaches usually teach aerial twisting in backward somersaults by means of basic sequences of arms rising/lowering in the frontal plane (i.e., abd/adduction: further referred to as \textit{2D techniques}), leaving room for improvement~\cite{bailly2020optimal}. 
% With practice and experience, athletes manage to refine their techniques by adding 3D deviations from the technique they were thought.


% Computer simulations are relevant tools for seeking innovative techniques in sports and especially in disciplines in which trials of new techniques increase the risk of injury, which is the case in trampoline. 
% Numerical optimization has been shown to be particularly useful for analyzing athletes motion and for discovering innovative techniques ~\cite{ashby2006optimal, spagele1999multi, yeadon2000mechanics}.
% Studies in which numerical optimization was used to investigate twist rotation in acrobatic sports using arms techniques (\cite{bharadwaj2016diver, dullin2016twisting, yeadon2017airborne, yeadon2017twist}) relied only on arms abd/adduction actions. 
% These aerial 2D arm techniques have been described in detail, leading to practical advices for coaches on efficient strategies to create twist rotations~\cite{yeadon_2015}.
% To our knowledge, no similar advises have been provided for 3D arm techniques.
% However, new techniques combining abd/adduction with changes in plane of elevation, referred to as \textit{3D techniques}, can increase the number of twists achievable~\cite{bailly2020optimal}.


% Despite their relevance, optimal simulations have an unfair advantage over athletes regarding accuracy.
% Indeed, even elite athletes experience variations in the execution of their skills from one trial to another~\cite{huchez2016differences}. 
% Due to this variability in their kinematics, athletes learn to make in-flight adjustments based on visual and vestibular feedbacks to land successfully~\cite{yeadon1996control}.
% Therefore, to make sure that the proposed optimal techniques can be performed by athletes, it is necessary to guarantee their robustness.
% For this purpose, robustness can be part of the optimal control problem (OCP) ~\cite{hiley2013investigating} or can be evaluated after optimization~\cite{mombaur2006performing}.
% Since experienced athletes adapt their kinematics to cope with perturbations, it is hypothesized that they use feedforward and feedback controls to modify their motor pattern~\cite{sayyah2017adjustment}.
% Indeed, athletes' responses to the sensory information are too complex to be reflexes~\cite{heinen2018spatial, balter2004habituation}.
% Therefore, it can be hypothesized that the internal processes aiming to determine the best response to perturbations can be modeled by optimization.

% The first aim of this study was to find innovative and robust twisting techniques. 
% A secondary objective was to provide a biomechanical explanation of the underlying strategies to explain why 3D techniques outperform 2D techniques.
% This explanation relies on a novel reference plane called the ``\textit{best tilting plane}'' (BTP).
% We hypothesized that for similar twist performance in 2D and 3D, 3D techniques would require less complex arm motions since arms can reach the BTP for a longer period of time.
% We also hypothesized that the number of twists achieved is correlated to arm motion complexity.
% These findings, combined with a robustness analysis and a thorough discussion of hand-picked salient solutions, will be of interest for acrobatic sports athletes and coaches.

To Do