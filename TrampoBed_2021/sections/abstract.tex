\begin{abstract}
% Aerial twisting techniques have gained coaches' interest since the flight time is part of the score in trampoline. 
% As these techniques are not intuitive, computer simulation has been a relevant tool to explore a variety of techniques. 
% Unfortunately, twisting somersaults were mainly simulated using arm abd/adduction only. 
% Our objective was to explore more complex but still anatomically feasible arm techniques to find innovative and robust twisting techniques.
% The twist rotation was maximized in a straight backward somersault performed by a model including abd/adduction with and without % change in plane of elevation. 
% This optimal control problem was solved by direct multiple-shooting. 
% A multi-start approach (n=440) was used to find a series of locally optimal performances. 
% The robustness of the selected techniques was assessed by adding noise on the arm kinematics and then reoptimizing to mimic the athlete's adaptation to kinematic changes. 
% Three innovative techniques which generate approximately three pure aerial twists were selected. 
% Techniques found by optimization share a highly twisting strategy consisting in moving the arm in a plane formed by the twisting and angular momentum axes.
% A mechanical analysis is presented to help coaches include this strategy in their professional practice. 
To Do
\end{abstract}
